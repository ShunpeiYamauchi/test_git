\documentclass{jsarticle}
\usepackage{amsmath}
\usepackage{amssymb}
\usepackage{cases}
\usepackage[dvips]{graphicx}
\usepackage{ascmac}
\usepackage{slashbox}
\usepackage{array}
\begin{document}
\section{変分原理とは何か}
局所的な記述:math::differential equations(ODE, PDE, SDE, etc.), physics::Newton eq, Maxwell eq, Schrodinger eq, etc)

大域的な記述:???

\subsection{言葉の準備}
・独立変数の変分

独立変数 $x$ が $x=a$ から $\tilde{x} = a+ \Delta x$ になったときの $\Delta x$ を独立変数 $x$ の変分と呼ぶ.

・従属変数(函数)の変分

独立変数 $x$ の変化に応じて函数 $f(x) = f(a)$ が $f(\tilde{x}) =f(a) +\Delta f$ になったときの $\Delta f$ を函数 $f$ の変分と呼ぶ.

\subsubsection{変分の性質と最大最小問題}
$x=a$ における変分 $\Delta f$ が $\Delta f \ge 0$ であれば, $f(x)$ は $x=a$ で最小.
\subsection{様々な函数のパラメーター}
\subsubsection{可算個のパラメーター}
$f(\{x_i\})$ の形でかける場合 $\Rightarrow$  $\Delta f = \sum_i \Delta_i f$ \quad ($\Delta_i f$ は $i$ 番目の変数 $x_i$ の変化に応じた変分)

\subsubsection{非可算個のパラメーターとオイラーの公式}
パラメーター $x$ を用いて $y=y(x), y'(x) =dy/dx$ により
\begin{equation}
I=\int_{x_0}^{x_1} f(x,y(x), y'(x)) dx
\end{equation}
の変分を考える.このときの $I$ のように,函数 $y(x), y'(x)$ を変数として定まるものを汎関数と呼ぶ.

$y(x)$ を $y(x) + \eta (x)$ だけ変化させたときの $I$ の変分 $\delta I$ は $\nu (x)$ の 1 次まで求めると,
\begin{eqnarray}
\delta I &= \int_{x_0}^{x_1} \left( f_y \cdot \eta (x) + f_{y'} \cdot \eta' (x)\right)\ dx  \nonumber \\
	   &= \int_{x_0}^{x_1} \left( f_y -\frac{d}{dx} f_{y'} \right) dx + \left[ f_{y'} \cdot \eta(x) \right]_{x_0}^{x_1}
\end{eqnarray}
となり,境界条件 $\eta(x_0) = \eta (x_1) =0$ のもとで $\delta I = 0$ となる十分条件は
\begin{equation}
f_y - \frac{d}{dx} f_{y'} =0
\end{equation} 
とかける.この等式をオイラーの公式という.また, $I$ を極値にするような函数 $y=y(x)$ を停留函数(stationary function)と呼ぶ.
\subsubsection{オイラーの公式の便利な変形}
(4) を用いて次のような等式が得られる.
\begin{equation}
\frac{d}{dx} \left( f -y'f_{y'} \right) = \frac{\partial f}{\partial x}
\end{equation}
導出の際にはオイラーの公式を $df_{y'}/dx = f_y$ に注意すると良い.
さらに, $f$ が陽に $x$ を含まない場合は,定数 $c$ を用いて
\begin{equation}
f-y'f_{y'} = c
\end{equation}
と書ける.
\subsection{制約条件つきの変分問題(ラグランジュの未定係数法)}
\begin{equation}
J=\int_{x_0}^{x_1} g(x,y,y')\ dx
\end{equation}
が一定の値をとる条件のもとで,
\begin{equation}
I=\int_{x_0}^{x_1} f(x,y,y')\ dx
\end{equation}
の最小値を求める.

$\Rightarrow$ $ I +\lambda J$ の極値を制約条件なしに調べれば良い($\lambda \in \mathbb{R}$)

\begin{equation}
\Rightarrow \frac{d}{dx} f_{y'} -f_y +\lambda \left( \frac{d}{dx} g_{y'} -g_y \right) = 0
\end{equation}

\underline{例題}

(ここに図)

斜線部の面積 $I$ を最小にする $y(x)$ は?

曲線の長さ $J$ は
\begin{equation}
J = \int_{x_0}^{x_1} \sqrt{1+(y')^2}\ dx,
\end{equation}
面積 $I$ は
\begin{equation}
I = \int_{x_0}^{x_1} y\ dx
\end{equation}
であり,双方 $x$ を陽に持たない函数の積分で表されている.従って,$I + \lambda J$ が極値をとる $y(x)$ はオイラーの公式から,定数 $c$ を用いて
\begin{equation}
y +\lambda \left( \sqrt{1+(y')^2}+ y'\cdot \frac{y'}{\sqrt{1+(y')^2}}  \right) = c
\end{equation}
を解くことにより与えられる.この式は容易に次のように変形できる.
\begin{equation}
y' = \pm \frac{\sqrt{\lambda^2 -(y-c)^2}}{y-c}
\end{equation}
この解はさらなる任意定数 $b$ を用いて
\begin{equation}
(x-b)^2 + (y-c)^2 = \lambda^2
\end{equation}
という形になる.各パラメータは$\lambda =J/\pi,\ b=(x_0+x_1)/2,\ c = 0, x_1 =x_0 +2\lambda$.

\underline{注}

 $I$ を固定,$J$ を最小にするためには $J +\lambda I$ を極小にすれば良い.この場合も円が停留曲線になる.
\end{document}